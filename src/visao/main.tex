\documentclass{report}

\usepackage[utf8]{inputenc}
% Pacote para manipular o espaçamento entre linhas
\usepackage[brazil]{babel}
% Pacote para manipular o espaçamento entre linhas
\usepackage{setspace}
% Aplica indentação ao primeiro parágrafo das seções
\usepackage{indentfirst}
% Modifica cores do documento
\usepackage[usenames, dvipsnames]{xcolor}
% Suporte à imagens
\usepackage{graphicx}
% Manipular posicionamento de figuras
\usepackage{float}
% Pacore para mesclar linhas em tabelas
\usepackage{multirow}
% Extensão ao bibtex, altera também o estilo das extensões
\usepackage{natbib}
% Suporte adicional a colunas
\usepackage{multicol}
% Suporte a hiperlinks
\usepackage{hyperref}
% Controle de tabelas avançado
\usepackage{tabularx}

\title{GEDU - Documento de Requisitos}
\author{José Silva}
\date{\today}

\begin{document}

\thispagestyle{empty}
\maketitle
\newpage

% Inicia a contagem de páginas do 1
\setcounter{page}{1}
% Alterar contagem das páginas iniciais para
% algarismos romanos
\pagenumbering{roman}
\tableofcontents
\newpage

% Gera lista de figuras
\listoffigures
\newpage

% Adicionando lista de tabelas
\listoftables
\newpage

% Volta para contagem normal
\setcounter{page}{1}
\pagenumbering{arabic}

\chapter{Introdução}

% retirado de: https://www.ibm.com/support/knowledgecenter/pt-br/SSWMEQ_4.0.6/com.ibm.rational.rrm.help.doc/topics/r_vision_doc.html
% TODO fazer introdução

Retirado de \href{https://www.ibm.com/support/knowledgecenter/pt-br/SSWMEQ_4.0.6/com.ibm.rational.rrm.help.doc/topics/r_vision_doc.html}{IBM Knowledge center}

Esta introdução fornece uma visão geral de todo o documento de visão. Ela
inclui o propósito, escopo, definições, acrônimos, abreviações, referências e
visão geral de todo o documento.

\section{Propósito}

Determina o propósito deste documento de visão.

\section{Escopo}

Descreve brevemente o escopo deste documento de visão, incluindo a quais
programas, projetos, aplicativos e processos de negócios o documento está
associado. Inclui qualquer outra coisa que este documento afete ou influencie.

\section{Definições, acrônimos e abreviações}

Define todos os termos, acrônimos e abreviações necessários para interpretar a
visão corretamente. Essas informações podem ser fornecidas por referência ao
glossário do projeto, que pode ser desenvolvido online no repositório do RM.

\section{Referências}

Referências: Lista todos os documentos aos quais o documento de visão faz
referência. Identifique cada documento por título, número de relatório (se
aplicável), data e organização de publicação. Especifique as origens a partir
das quais os leitores podem obter as referências; as origens estão disponíveis
de maneira ideal no RM ou em outros repositórios online. Essas informações
podem ser fornecidas por referência para um apêndice ou para outro documento.

\section{Visão geral}

Visão geral: Descreve o conteúdo do documento de visão e explica como o
documento é organizado.

\chapter{Posicionando}

\section{Oportunidade de Negócios}

Descreve brevemente a oportunidade de negócios que é tratada por este projeto.

\section{Instrução do Problema}

Resume o problema que este projeto resolve. Use as seguintes instruções como um
modelo, fornecendo detalhes do projeto para substituir os elementos entre
parênteses:

O problema de (descreva o problema) afeta (as partes interessadas afetadas pelo
problema). O impacto do problema é (qual é o impacto do problema). Uma solução
bem sucedida incluiria (lista alguns principais benefícios de uma solução bem
sucedida).

\section{Instrução de Posição do Produto}

Fornece uma instrução geral resumida no nível mais alto, a posição exclusiva
que o produto pretende preencher no mercado de trabalho. Use as seguintes
instruções como um modelo, fornecendo detalhes do projeto para substituir os
elementos entre parênteses:

Para o (cliente alvo) quem (instrução da necessidade ou oportunidade). O (nome
do produto) é uma (categoria do produto) que (instrução do principal benefício,
isto é, o motivo convincente para comprar). De outro modo (principal
alternativa competitiva), nosso produto (instrução da principal diferenciação).

Uma instrução de posição do produto comunica o intento do aplicativo e a
importância do projeto para todas as partes interessadas.

\chapter{Descrições da Parte Interessada e do Usuário}

Para fornecer produtos e serviços que atendam às necessidades das partes
interessadas e dos usuários, você deve identificar e envolver todas as partes
interessadas como parte do processo de definição dos requisitos. Você deve
também identificar os usuários do sistema e assegurar que a comunidade das
partes interessadas os represente adequadamente.

Esta seção fornece um perfil das partes interessadas e usuários que estão
envolvidos no projeto. Esta seção também identifica os principais problemas que
as partes interessadas e os usuários consideram que a solução proposta deva
tratar. Esta seção não descreve as solicitações ou requisitos específicos; um
artefato separado de solicitações da parte interessada captura esses itens. A
descrição do principal problema fornece o plano de fundo e a justificação para
os requisitos.

\section{Demográficos de Mercado}

Resume os principais demográficos de mercado que motivam suas decisões sobre o
produto. Descrevem e posicionam os segmentos do mercado alvo. Estime o tamanho
e o crescimento do mercado usando o número de usuários potenciais. Como
alternativa, estime a quantia de dinheiro que seus clientes gastam tentando
atender às necessidades que seu produto ou aprimoramento preencheria. Revise as
principais tendências do segmento de mercado e tecnologias. Responda estas
questões estratégicas: Qual é a reputação de sua organização nesses mercados? O
que você gostaria que a reputação fosse? Como esse produto ou serviço suporta
seus objetivos? \section{Resumo da Parte Interessada}

Lista todas as partes interessadas identificadas. Para cada tipo de parte
interessada, forneça estas informações:

\begin{enumerate}

	\item
	      \textbf{Nome:}Nome do tipo da parte interessada.

	\item
	      \textbf{Representa: }Descreve brevemente quais pessoas, equipes ou organizações
	      esse tipo de parte interessada representa.

	\item
	      \textbf{Função:} Descreve brevemente a função que esse tipo de parte
	      interessada desempenha no esforço de desenvolvimento.

\end{enumerate}

\section{Resumo do Usuário}

Lista todos os tipos de usuários identificados. Para cada tipo de usuário,
forneça estas informações:

\begin{enumerate}

	\item
	      Nome: Nome do tipo de usuário

	\item
	      Descrição: Descreve brevemente o relacionamento desse tipo de usuário com o
	      sistema que está em desenvolvimento.

	\item
	      Parte Interessada: Lista qual tipo de parte interessada representa esse tipo de
	      usuário.

\end{enumerate}

\section{Ambiente do Usuário}

Detalha o ambiente de trabalho do usuário alvo. Aqui estão algumas sugestões:

\begin{itemize}

	\item
	      Quantas pessoas estão envolvidas na conclusão da tarefa? Está sendo alterado?

	\item
	      Quanto tempo leva um loop de tarefa? Quanto tempo os usuários gastam em cada
	      atividade? Está sendo alterado?

	\item
	      Quais restrições de ambiente exclusivas afetam o projeto? Por exemplo, os
	      usuários requerem dispositivos remotos, trabalham externamente ou trabalham
	      durante as viagens?

	\item
	      Quais plataformas de sistema estão em uso atualmente? Existem plataformas
	      futuras planejadas?

	      Que outros aplicativos estão em uso? Seu aplicativo precisa se integrar a eles?

\end{itemize}

Nesta seção, você pode incluir extrações do modelo de negócio para descrever a
tarefa e os trabalhadores envolvidos.

\section{Perfis das Partes Interessadas}

Descreve cada parte interessada no projeto, preenchendo a seguinte tabela para
cada parte interessada. Lembre-se: os tipos de partes interessadas podem ser
usuários, departamentos estratégicos, departamentos jurídicos ou de
conformidade, desenvolvedores técnicos, equipes de operação, entre outros. Um
perfil completo abrange os seguintes tópicos para cada tipo de parte
interessada:

\begin{itemize}

	\item
	      \textbf{Representante:} Determina quem representa a parte interessada para o
	      projeto (Essa informação será opcional se estiver documentada em algum outro
	      lugar.) Insira os nomes dos representantes.

	\item
	      \textbf{Descrição:} Descreve brevemente o tipo de parte interessada.

	\item
	      \textbf{Tipo:} Qualifica o conhecimento da parte interessada, como
	      "usuário avançado", "especialista em negócios", ou "usuário informal". Essa
	      designação pode sugerir a experiência técnica e o grau de sofisticação.

	\item
	      \textbf{Responsabilidades:} Lista as principais responsabilidades da parte
	      interessada no sistema em desenvolvimento; lista seus interesses como uma parte
	      interessada.

	\item
	      \textbf{Critérios de Sucesso:} Determina como a parte interessada define o sucesso. Como
	      a parte interessada é recompensada?

	\item
	      \textbf{Envolvimento:} Descreve como a parte interessada está envolvida no
	      projeto. Onde possível, relate o envolvimento nas funções do processo; por
	      exemplo, uma parte interessada pode ser um revisor de requisitos.

	\item
	      \textbf{Entregas:} Identifica as entregas adicionais que a parte
	      interessada requer. Esses itens podem ser entregas do projeto ou saída a partir
	      do sistema em desenvolvimento.

	\item
	      \textbf{Comentários ou Problemas:} Determina os problemas que interferem com o sucesso e
	      quaisquer outras informações relevantes.

\end{itemize}

\section{Perfis do Usuário}

Descreve cada usuário do sistema aqui, preenchendo a seguinte tabela para cada
tipo de usuário. Lembre-se que os tipos de usuário podem ser especialistas e
novatos; por exemplo, um especialista pode precisar de uma ferramenta
sofisticada e flexível com suporte para várias plataformas, enquanto um novato
pode precisar de uma ferramenta que seja fácil de usar. Um perfil completo
abrange esses tópicos para cada tipo de usuário:

\begin{itemize}

	\item
	      \textbf{Representante:} Indica quem representa o usuário para o projeto. (Essa
	      informação será opcional se estiver documentada em algum outro lugar.) Esse
	      representante, geralmente refere-se à parte interessada que representa o
	      conjunto de usuários; por exemplo, Parte Interessada: Parte Interessada1.

	\item
	      \textbf{Descrição:} Descreve brevemente o tipo de usuário.

	\item
	      \textbf{Tipo:} Qualifica o conhecimento do usuário, como "usuário
	      avançado" ou "usuário informal." Essa designação pode sugerir a experiência
	      técnica e o grau de sofisticação.

	\item
	      \textbf{Responsabilidades:} Lista as principais responsabilidades do usuário com
	      respeito ao sistema; por exemplo, determina quem captura os detalhes do
	      cliente, produz relatórios e coordena trabalho, etc.

	\item
	      \textbf{Critérios de Sucesso:} Determina como o usuário define o sucesso. Como o
	      usuário é recompensado?

	\item
	      \textbf{Envolvimento:} Descreve como o usuário está envolvido no projeto. Onde
	      possível, relate o envolvimento nas funções do processo; por exemplo, uma parte
	      interessada pode ser um revisor de requisitos.

	\item
	      \textbf{Entregas:} Identifica as entregas que o usuário produz e para quem.

	\item
	      \textbf{Comentários ou Problemas:} Determina os problemas que interferem com o sucesso e
	      quaisquer outras informações relevantes. Descreve as tendências que tornam a
	      tarefa do usuário mais fácil ou mais difícil.

\end{itemize}

\section{Principais Necessidades da Parte Interessada ou do Usuário}

Lista os principais problemas com soluções existentes como observadas pela
parte interessada. Esclarece estas questões para cada problema:

\begin{itemize}

	\item
	      Quais são os motivos para esse problema?

	\item
	      Como o problema é resolvido agora?

	\item
	      Quais soluções a parte interessada deseja?

\end{itemize}

Você deve entender a importância relativa que a parte interessada coloca na
solução de cada problema. A classificação e técnicas de votos acumulativos
indicam os problemas que devem ser resolvidos versus os problemas que as partes
interessadas gostariam de tratar. Use esta tabela para capturar as necessidades
da parte interessada.

\begin{table}[H]
	\centering

	\begin{tabularx}{\textwidth}{X|X|X|X|X}
		\textbf{Necessidade} & \textbf{Prioridade} & \textbf{Interesses} & \textbf{Solução atual} & \textbf{Solução proposta} \\
		\hline
		                     &                     &                     &                        &                           \\
		\hline
	\end{tabularx}

	\caption[Tabela de necessidades]{Tabela de necessidades a serem definidas}

	\label{tab:necessidades}

\end{table}

\section{Alternativas e Concorrência}

Identifica as alternativas que a parte interessada percebe como disponíveis.
Essas alternativas podem incluir a compra do produto de um concorrente, a
criação de uma solução desenvolvida internamente ou manter o status quo. Listam
todas as opções disponíveis e conhecidas. Elas incluem os principais pontos
fortes e fracos de cada concorrente como observados pela parte interessada.

\chapter{Visão geral do produto}

Esta seção fornece uma visualização de alto nível das capacidades do produto,
interfaces para outros aplicativos e configurações dos sistemas. Esta seção, em
geral, consiste em três subseções:

\begin{itemize}

	\item
	      Perspectiva do Produto

	\item
	      Funções do Produto

	\item
	      Suposições e Dependências

\end{itemize}

\section{Perspectiva do Produto}

Coloca o produto em perspectiva com respeito a outros produtos relacionados e
ao ambiente do usuário. Se o produto for independente e totalmente autocontido,
indique-o aqui. Se o produto for um componente de um sistema maior, relacione
como esses sistemas interagem e identificam as interfaces relevantes entre os
sistemas. Uma maneira de exibir os principais componentes do maior sistema,
interconexões e interfaces externas é usar um processo de negócios ou diagrama
de casos de uso.

\section{Resumo das Capacidades}

Resume os principais benefícios e recursos que o produto fornecerá. Por
exemplo, um sistema de suporte ao cliente pode usar essa parte para endereçar a
documentação do problema, o roteamento e o relato de status sem elaborar em
detalhes o que essas funções requerem. Organize as funções de modo que a lista
seja compreensível para o cliente ou para qualquer outra pessoa que leia o
documento pela primeira vez. Uma simples tabela que lista os principais
benefícios cujos recursos de suporte são suficientes, como no exemplo a seguir.

\begin{table}[H]
	\centering

	\begin{tabularx}{\textwidth}{X|X}
		\textbf{Benefício para o Cliente}                                                                & \textbf{Recursos de suporte}                                                                                                                                                           \\
		\hline
		A nova equipe de suporte pode aprender rapidamente como usar o produto.                          & A base de conhecimento ajuda a equipe de suporte a identificar rapidamente as correções e soluções alternativas conhecidas.                                                            \\
		A satisfação do cliente é melhorada porque não há falhas.                                        & Os problemas são exclusivamente detalhados em itens, classificados e controlados em todo o processo de resolução. A notificação automática ocorre para quaisquer problemas anteriores. \\
		O gerenciamento pode identificar as áreas com problema e calibrar a carga de trabalho da equipe. & Os relatórios de tendência e distribuição permitem a revisão de alto nível do status do problema.                                                                                      \\
		Equipes de suporte distribuídas podem trabalhar juntas para resolver problemas.                  & Com um servidor de replicação, as informações do banco de dados podem ser compartilhadas em toda a empresa.                                                                            \\
		Os clientes podem se ajudar, reduzindo os custos de suporte e melhorando o tempo de resposta.    & A base de conhecimento pode ficar disponível na Internet. A base de conhecimento inclui recursos de pesquisa de hipertexto e um mecanismo de consulta gráfica.                         \\
		\hline
	\end{tabularx}

	\caption[Benefícios e Recursos]{Tabela contendo benefícios e recursos do produto}
	\label{tab:beneficios}

\end{table}

\section{Suposições e Dependências}

Lista cada um dos fatores que afeta os recursos que o documento de visão
inclui. Lista as suposições que, se modificadas, alterarão o documento de
visão. Por exemplo, uma suposição pode indicar que um sistema operacional
específico fique disponível para o hardware designado para o produto de
software. Se o sistema operacional não estiver disponível, será necessário
alterar o documento de visão.

\section{Custo e Precificação}

Registra os impactos e restrições relevantes de custo e precificação. Por
exemplo, os custos de distribuição (o número de CDs e CD principal) ou outras
restrições de custo de mercadorias vendidas (manuais e embalagem) podem ser
material para o sucesso dos projetos, ou irrelevantes, dependendo da natureza
do aplicativo.

\section{Licenciamento e Instalação}

Os problemas de licenciamento e instalação também podem impactar diretamente o
esforço de desenvolvimento. Por exemplo, a necessidade de suportar a
serialização, a segurança da senha ou o licenciamento da rede criarão
requisitos adicionais do sistema que devem ser considerados no esforço de
desenvolvimento. Os requisitos de instalação também podem afetar a codificação
ou criar a necessidade de separar o software de instalação.

\chapter{Recursos do Produto}

Lista e descreve brevemente os recursos do produto. Os recursos são capacidades
de alto nível do sistema que são necessários para entregar benefícios aos
usuários. Cada recurso é um serviço solicitado que, em geral, requer uma série
de entradas para alcançar o resultado desejado. Por exemplo, um recurso de um
sistema de rastreamento de problemas pode ser a capacidade de fornecer
relatórios de tendências. À medida que o modelo de casos de uso toma forma,
atualize a descrição para fazer referência aos casos de uso.

Como o documento de visão é revisado por uma ampla variedade de equipes
envolvidas, mantenha o nível de detalhes gerais suficiente para que todos
possam entender. No entanto, ofereça detalhes suficientes para fornecer à
equipe as informações que ela precisa para criar um modelo de casos de uso ou
outros documentos de design.

Para gerenciar a complexidade do aplicativo, para um novo sistema ou uma
mudança incremental, liste os recursos em um alto nível para que você possa
incluir aproximadamente 25 a 99 recursos. Esses recursos fornecem a base para a
definição do produto, gerenciamento de escopo e gerenciamento do projeto. Cada
recurso será expandido mais detalhadamente no modelo de casos de uso.

Em toda esta seção, torne cada recurso relevante para usuários, operadores ou
outros sistemas externos. Inclua uma descrição de funções e problemas de
usabilidade que devem ser tratados. As seguintes diretrizes se aplicam:

\begin{itemize}

	\item
	      Evite design. Mantenha as descrições do recurso em um nível geral. Foque nas
	      capacidades necessárias e por que (não como) elas devem ser implementadas.

	\item
	      Designe todos os recursos como requisitos de um tipo de recurso específico para
	      fácil referência e rastreamento.

\end{itemize}

\chapter{Restrições}

Observe todas as restrições de design, restrições externas, como requisitos
operacionais ou regulamentares) ou outras dependências.

\chapter{Faixas de qualidade}

Defina as faixas de qualidade para desempenho, robustez, tolerância a falhas,
usabilidade e características similares que o conjunto de recursos não
descreve.

\chapter{Precedência e Prioridade}

Define a prioridade dos diferentes recursos do sistema.

\chapter{Outros requisitos do produto}

Em um alto nível, lista os padrões aplicáveis, os requisitos de hardware ou
plataforma, os requisitos de desempenho e os requisitos ambientais.

\section{Padrões Aplicáveis}

Lista todos os padrões que o produto deve estar em conformidade. A lista pode
incluir estes padrões:

\begin{itemize}
	\item
	      Padrões jurídicos e regulamentares (FDA, UCC)

	\item
	      Padrões de comunicações (TCP/IP, ISDN)

	\item
	      Padrões de conformidade da plataforma (Windows, UNIX, etc.)

	\item
	      Padrões de qualidade e segurança (UL, ISO, CMM)
\end{itemize}

\section{Requisitos do Sistema}

Define os requisitos do sistema para o aplicativo. Eles incluem os sistemas
operacionais do host suportados e as plataformas de rede, configurações,
memória, dispositivos periféricos e software de parceiros.

\section{Requisitos de Desempenho}

Detalha os requisitos de desempenho. Os problemas de desempenho podem incluir
itens como fatores de carga do usuário, largura de banda ou capacidade de
comunicação, rendimento, exatidão, confiabilidade ou tempos de resposta em uma
variedade de condições de carregamento.

\section{Requisitos Ambientais}

Detalha os requisitos ambientais conforme necessário. Para sistemas baseados em
hardware, os problemas ambientais podem incluir temperatura, choque elétrico,
umidade e radiação. Para aplicativos de software, os fatores ambientais podem
incluir condições de uso, ambiente do usuário, disponibilidade do recurso,
problemas de manutenção, manipulação de erros e recuperação.

\chapter{Requisitos de Documentação}

Esta seção descreve a documentação que deve ser desenvolvida para suportar a
implementação bem sucedida do aplicativo.

\section{Notas sobre a liberação, arquivo Leia-me}

As notas sobre a liberação ou um arquivo Leia-me abreviado podem incluir uma
seção "O que Há de Novo", uma discussão sobre problemas de compatibilidade com
liberações anteriores, e alertas de instalação e atualização. O documento pode
também conter ou vincular correções na liberação e quaisquer problemas ou
soluções alternativas conhecidos. \section{Ajuda On-line}

Muitos aplicativos fornecem um sistema de ajuda on-line para ajudar o usuário.
A natureza desses sistemas é exclusiva para desenvolvimento de aplicativo, pois
eles combinam aspectos de programação (centros de informações pesquisáveis e
navegação do tipo Web) com aspectos de composição técnica (organização,
apresentação). Muitas equipes consideram que o desenvolvimento do sistema de
ajuda on-line é um projeto dentro de um projeto que se beneficia do
gerenciamento de escopo e planejamento no início do projeto.

\section{Guias de Instalação}

Um documento que inclui instalação, configuração e instruções de atualização
como parte da oferta de solução integral.

\section{Rótulo e Embalagem}

Uma aparência consistente começa com a embalagem do produto e se aplica aos
menus de instalação, telas iniciais, sistemas de ajuda, caixas de diálogo de
GUI e assim por diante. Esta seção define as necessidades e tipos de rótulos a
serem incorporados no código. Os exemplos incluem copyright e avisos de
patentes, logotipos corporativos, ícones padronizados e outros elementos
gráficos.

\chapter{Apêndice 1 - Atributos do Recurso}

Fornece aos recursos atributos que podem ser usados para avaliar, controlar,
priorizar e gerenciar os itens de produtos propostos para implementação.
Descreve todos os tipos de requisitos e atributos em um plano de gerenciamento
de requisitos. No entanto, você pode listar e descrever brevemente os atributos
para os recursos que foram escolhidos. As subseções a seguir representam um
conjunto de atributos de recursos sugeridos.

\section{Status}

As equipes configuram o status do recurso após negociação e revisão pela equipe
de gerenciamento do projeto. O status controla o progresso enquanto durar o
projeto. A tabela a seguir fornece um exemplo de valores típicos do atributo de
status.

\begin{table}[H]
	\centering

	\begin{tabularx}{\textwidth}{X|X}
		\textbf{Status} & \textbf{Descrição}                                                                                                                                                                                                                                                     \\
		\hline
		Proposta        & Descreve os recursos que estão em discussão, mas não foram revistos e aceitos pelo "canal oficial", O canal oficial pode ser um grupo de trabalho que consiste em representantes da equipe do projeto, gerenciamento do produto e comunidade do usuário ou do cliente. \\
		Aprovado        & Capacidades que são consideradas úteis e factíveis e que foram aprovadas para implementação pelo canal oficial.                                                                                                                                                        \\
		Incorporado     & Os recursos que foram incorporados na linha de base do produto.                                                                                                                                                                                                        \\
		\hline
	\end{tabularx}

	\caption[Valores de status]{Exemplos de valores de status}
	\label{tab:status}
\end{table}

\section{Benefício}

O grupo de marketing, o gerente de produto ou o analista de negócios configura
os benefícios do recurso. Todos os requisitos não são criados igualmente. A
classificação de requisitos por seu benefício relativo para o usuário final
abre um diálogo com clientes, analistas e membros da equipe de desenvolvimento.
Use os benefícios no gerenciamento de escopo e na determinação da prioridade de
desenvolvimento. A tabela a seguir fornece um exemplo de valores de atributos
típicos de benefício ou prioridade.

\begin{table}[H]
	\centering

	\begin{tabularx}{\textwidth}{X|X}
		\textbf{Status} & \textbf{Descrição}                                                                                                                                                                                                                                                                                     \\
		\hline
		Crítico         & Recursos essenciais. A falha na implementação de um recurso crítico significa que o sistema não atenderá às necessidades do cliente. Todos os recursos críticos devem ser implementados na liberação ou a programação falhará.                                                                         \\
		Importante      & Recursos essenciais. A falha na implementação de um recurso crítico significa que o sistema não atenderá às necessidades do cliente. Todos os recursos críticos devem ser implementados na liberação ou a programação falhará.                                                                         \\
		Útil            & Os recursos que são úteis em menos aplicativos típicos, são usados menos frequentemente, ou podem corresponder às soluções alternativas razoavelmente eficientes. Nenhuma receita significativa ou impacto na satisfação do cliente poderá ser esperada se tal item não for incluído em uma liberação. \\
		\hline
	\end{tabularx}

	\caption[Prioridades de benefícios]{Exemplos de prioridades de benefícios}
	\label{tab:prioridades}
\end{table}

\section{Esforço}

A equipe de desenvolvimento estima o esforço necessário para implementar os
recursos. Alguns recursos requerem mais tempo e recursos do que outros. A
estimativa de tempo, código necessário ou funções, ajuda a calibrar a
complexidade e definir expectativas do que pode ser realizado em um determinado
período de tempo. Use a estimativa para gerenciar o escopo e determinar a
prioridade de desenvolvimento.

\section{Risco}

A equipe de desenvolvimento estabelece os níveis de risco, com base na
probabilidade do projeto experimentar eventos indesejados, como saturações de
custos, atrasos na programação ou até cancelamento. A maioria dos gerentes de
projetos consideram os riscos de categorização como alto, médio e
suficientemente baixo, embora as gradações mais finas sejam possíveis. Em
geral, o risco pode ser avaliado indiretamente medindo-se a incerteza (faixa)
da estimativa de programação da equipes de projetos.

\section{Estabilidade}

O analista e a equipe de desenvolvimento estabelece a estabilidade do recurso
com base na probabilidade do recurso ser alterado ou no entendimento da equipe
do recurso ser alterado. A estabilidade é usada para ajudar a estabelecer
prioridades de desenvolvimento e determinar esses itens para os quais a
descoberta adicional é a próxima ação apropriada.

\section{Liberação de destino}

As equipes registram a versão anterior do produto pretendido que incluirá o
recurso. É possível usar esse campo para alocar recursos de um documento de
visão para um release de base de linha específico. Quando combinado com o campo
de status, sua equipe pode propor, registrar e discutir vários recursos da
liberação sem comprometer o desenvolvimento deles. Somente os recursos cujo
status é definido como "incorporado" e cuja liberação de destino é definida
serão implementados. Com o gerenciamento de escopo, o número da versão da
liberação de destino poderá ser aumentado para que o item permaneça no
documento de visão, mas seja programado para uma liberação posterior.

\section{Designado Para}

Na maioria dos projetos, os recursos serão designados para equipes de recursos
que são responsáveis pela descoberta adicional, compondo os requisitos e a
implementação do software. O processo ajuda todos na equipe do projeto a
entenderem melhor as responsabilidades.

\section{Motivo}

As equipes usam esse campo de texto para controlar a origem do recurso
solicitado. Os requisitos existem por motivos específicos. Esse campo registra
uma explicação ou uma referência a uma explicação. Por exemplo, a referência
pode apontar para uma página e número da linha de uma especificação de
requisito do produto, ou apontar para um marcador de minuto no vídeo de
entrevista de um cliente importante.

\end{document}
