\documentclass{report}

\usepackage[utf8]{inputenc}
% Pacote para manipular o espaçamento entre linhas
\usepackage[brazil]{babel}
% Pacote para manipular o espaçamento entre linhas
\usepackage{setspace}
% Aplica indentação ao primeiro parágrafo das seções
\usepackage{indentfirst}
% Modifica cores do documento
\usepackage[usenames, dvipsnames]{xcolor}
% Suporte à imagens
\usepackage{graphicx}
% Manipular posicionamento de figuras
\usepackage{float}
% Pacore para mesclar linhas em tabelas
\usepackage{multirow}
% Extensão ao bibtex, altera também o estilo das extensões
\usepackage{natbib}
% Suporte adicional a colunas
\usepackage{multicol}

\title{GEDU - Documento de Requisitos}
\author{José Silva}
\date{\today}

\begin{document}

\thispagestyle{empty}
\maketitle
\newpage

% Inicia a contagem de páginas do 1
\setcounter{page}{1}
% Alterar contagem das páginas iniciais para
% algarismos romanos
\pagenumbering{roman}
\tableofcontents
\newpage

% Gera lista de figuras
\listoffigures
\newpage

% Adicionando lista de tabelas
\listoftables
\newpage

% Volta para contagem normal
\setcounter{page}{1}
\pagenumbering{arabic}

\chapter{Introdução}

% TODO fazer introdução

Esta introdução fornece uma visão geral de todo o documento de visão. Ela
inclui o propósito, escopo, definições, acrônimos, abreviações, referências e
visão geral de todo o documento.

\begin{enumerate}

	\item
	      Propósito: Determina o propósito deste documento de visão.

	\item
	      Escopo: Descreve brevemente o escopo deste documento de visão, incluindo a
	      quais programas, projetos, aplicativos e processos de negócios o documento está
	      associado. Inclui qualquer outra coisa que este documento afete ou influencie.

	\item
	      Definições, acrônimos e abreviações: Define todos os termos, acrônimos e
	      abreviações necessários para interpretar a visão corretamente. Essas
	      informações podem ser fornecidas por referência ao glossário do projeto, que
	      pode ser desenvolvido online no repositório do RM.

	\item
	      Referências: Lista todos os documentos aos quais o documento de visão faz
	      referência. Identifique cada documento por título, número de relatório (se
	      aplicável), data e organização de publicação. Especifique as origens a partir
	      das quais os leitores podem obter as referências; as origens estão disponíveis
	      de maneira ideal no RM ou em outros repositórios online. Essas informações
	      podem ser fornecidas por referência para um apêndice ou para outro documento.

	\item
	      Visão geral: Descreve o conteúdo do documento de visão e explica como o
	      documento é organizado.
\end{enumerate}

\end{document}
